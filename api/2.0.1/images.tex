\batchmode
\documentclass{howto}
\RequirePackage{ifthen}




\title{Python Cryptography Toolkit}


\release{2.0.1}


\author{A.M. Kuchling}
\authoraddress{\url{www.amk.ca}}




\usepackage[dvips]{color}


\pagecolor[gray]{.7}

\usepackage[]{inputenc}



\makeatletter

\makeatletter
\count@=\the\catcode`\_ \catcode`\_=8 
\newenvironment{tex2html_wrap}{}{}%
\catcode`\<=12\catcode`\_=\count@
\newcommand{\providedcommand}[1]{\expandafter\providecommand\csname #1\endcsname}%
\newcommand{\renewedcommand}[1]{\expandafter\providecommand\csname #1\endcsname{}%
  \expandafter\renewcommand\csname #1\endcsname}%
\newcommand{\newedenvironment}[1]{\newenvironment{#1}{}{}\renewenvironment{#1}}%
\let\newedcommand\renewedcommand
\let\renewedenvironment\newedenvironment
\makeatother
\let\mathon=$
\let\mathoff=$
\ifx\AtBeginDocument\undefined \newcommand{\AtBeginDocument}[1]{}\fi
\newbox\sizebox
\setlength{\hoffset}{0pt}\setlength{\voffset}{0pt}
\addtolength{\textheight}{\footskip}\setlength{\footskip}{0pt}
\addtolength{\textheight}{\topmargin}\setlength{\topmargin}{0pt}
\addtolength{\textheight}{\headheight}\setlength{\headheight}{0pt}
\addtolength{\textheight}{\headsep}\setlength{\headsep}{0pt}
\setlength{\textwidth}{349pt}
\newwrite\lthtmlwrite
\makeatletter
\let\realnormalsize=\normalsize
\global\topskip=2sp
\def\preveqno{}\let\real@float=\@float \let\realend@float=\end@float
\def\@float{\let\@savefreelist\@freelist\real@float}
\def\liih@math{\ifmmode$\else\bad@math\fi}
\def\end@float{\realend@float\global\let\@freelist\@savefreelist}
\let\real@dbflt=\@dbflt \let\end@dblfloat=\end@float
\let\@largefloatcheck=\relax
\let\if@boxedmulticols=\iftrue
\def\@dbflt{\let\@savefreelist\@freelist\real@dbflt}
\def\adjustnormalsize{\def\normalsize{\mathsurround=0pt \realnormalsize
 \parindent=0pt\abovedisplayskip=0pt\belowdisplayskip=0pt}%
 \def\phantompar{\csname par\endcsname}\normalsize}%
\def\lthtmltypeout#1{{\let\protect\string \immediate\write\lthtmlwrite{#1}}}%
\newcommand\lthtmlhboxmathA{\adjustnormalsize\setbox\sizebox=\hbox\bgroup\kern.05em }%
\newcommand\lthtmlhboxmathB{\adjustnormalsize\setbox\sizebox=\hbox to\hsize\bgroup\hfill }%
\newcommand\lthtmlvboxmathA{\adjustnormalsize\setbox\sizebox=\vbox\bgroup %
 \let\ifinner=\iffalse \let\)\liih@math }%
\newcommand\lthtmlboxmathZ{\@next\next\@currlist{}{\def\next{\voidb@x}}%
 \expandafter\box\next\egroup}%
\newcommand\lthtmlmathtype[1]{\gdef\lthtmlmathenv{#1}}%
\newcommand\lthtmllogmath{\dimen0\ht\sizebox \advance\dimen0\dp\sizebox
  \ifdim\dimen0>.95\vsize
   \lthtmltypeout{%
*** image for \lthtmlmathenv\space is too tall at \the\dimen0, reducing to .95 vsize ***}%
   \ht\sizebox.95\vsize \dp\sizebox\z@ \fi
  \lthtmltypeout{l2hSize %
:\lthtmlmathenv:\the\ht\sizebox::\the\dp\sizebox::\the\wd\sizebox.\preveqno}}%
\newcommand\lthtmlfigureA[1]{\let\@savefreelist\@freelist
       \lthtmlmathtype{#1}\lthtmlvboxmathA}%
\newcommand\lthtmlpictureA{\bgroup\catcode`\_=8 \lthtmlpictureB}%
\newcommand\lthtmlpictureB[1]{\lthtmlmathtype{#1}\egroup
       \let\@savefreelist\@freelist \lthtmlhboxmathB}%
\newcommand\lthtmlpictureZ[1]{\hfill\lthtmlfigureZ}%
\newcommand\lthtmlfigureZ{\lthtmlboxmathZ\lthtmllogmath\copy\sizebox
       \global\let\@freelist\@savefreelist}%
\newcommand\lthtmldisplayA{\bgroup\catcode`\_=8 \lthtmldisplayAi}%
\newcommand\lthtmldisplayAi[1]{\lthtmlmathtype{#1}\egroup\lthtmlvboxmathA}%
\newcommand\lthtmldisplayB[1]{\edef\preveqno{(\theequation)}%
  \lthtmldisplayA{#1}\let\@eqnnum\relax}%
\newcommand\lthtmldisplayZ{\lthtmlboxmathZ\lthtmllogmath\lthtmlsetmath}%
\newcommand\lthtmlinlinemathA{\bgroup\catcode`\_=8 \lthtmlinlinemathB}
\newcommand\lthtmlinlinemathB[1]{\lthtmlmathtype{#1}\egroup\lthtmlhboxmathA
  \vrule height1.5ex width0pt }%
\newcommand\lthtmlinlineA{\bgroup\catcode`\_=8 \lthtmlinlineB}%
\newcommand\lthtmlinlineB[1]{\lthtmlmathtype{#1}\egroup\lthtmlhboxmathA}%
\newcommand\lthtmlinlineZ{\egroup\expandafter\ifdim\dp\sizebox>0pt %
  \expandafter\centerinlinemath\fi\lthtmllogmath\lthtmlsetinline}
\newcommand\lthtmlinlinemathZ{\egroup\expandafter\ifdim\dp\sizebox>0pt %
  \expandafter\centerinlinemath\fi\lthtmllogmath\lthtmlsetmath}
\newcommand\lthtmlindisplaymathZ{\egroup %
  \centerinlinemath\lthtmllogmath\lthtmlsetmath}
\def\lthtmlsetinline{\hbox{\vrule width.1em \vtop{\vbox{%
  \kern.1em\copy\sizebox}\ifdim\dp\sizebox>0pt\kern.1em\else\kern.3pt\fi
  \ifdim\hsize>\wd\sizebox \hrule depth1pt\fi}}}
\def\lthtmlsetmath{\hbox{\vrule width.1em\kern-.05em\vtop{\vbox{%
  \kern.1em\kern0.8 pt\hbox{\hglue.17em\copy\sizebox\hglue0.8 pt}}\kern.3pt%
  \ifdim\dp\sizebox>0pt\kern.1em\fi \kern0.8 pt%
  \ifdim\hsize>\wd\sizebox \hrule depth1pt\fi}}}
\def\centerinlinemath{%
  \dimen1=\ifdim\ht\sizebox<\dp\sizebox \dp\sizebox\else\ht\sizebox\fi
  \advance\dimen1by.5pt \vrule width0pt height\dimen1 depth\dimen1 
 \dp\sizebox=\dimen1\ht\sizebox=\dimen1\relax}

\def\lthtmlcheckvsize{\ifdim\ht\sizebox<\vsize 
  \ifdim\wd\sizebox<\hsize\expandafter\hfill\fi \expandafter\vfill
  \else\expandafter\vss\fi}%
\providecommand{\selectlanguage}[1]{}%
\makeatletter \tracingstats = 1 


\begin{document}
\pagestyle{empty}\thispagestyle{empty}\lthtmltypeout{}%
\lthtmltypeout{latex2htmlLength hsize=\the\hsize}\lthtmltypeout{}%
\lthtmltypeout{latex2htmlLength vsize=\the\vsize}\lthtmltypeout{}%
\lthtmltypeout{latex2htmlLength hoffset=\the\hoffset}\lthtmltypeout{}%
\lthtmltypeout{latex2htmlLength voffset=\the\voffset}\lthtmltypeout{}%
\lthtmltypeout{latex2htmlLength topmargin=\the\topmargin}\lthtmltypeout{}%
\lthtmltypeout{latex2htmlLength topskip=\the\topskip}\lthtmltypeout{}%
\lthtmltypeout{latex2htmlLength headheight=\the\headheight}\lthtmltypeout{}%
\lthtmltypeout{latex2htmlLength headsep=\the\headsep}\lthtmltypeout{}%
\lthtmltypeout{latex2htmlLength parskip=\the\parskip}\lthtmltypeout{}%
\lthtmltypeout{latex2htmlLength oddsidemargin=\the\oddsidemargin}\lthtmltypeout{}%
\makeatletter
\if@twoside\lthtmltypeout{latex2htmlLength evensidemargin=\the\evensidemargin}%
\else\lthtmltypeout{latex2htmlLength evensidemargin=\the\oddsidemargin}\fi%
\lthtmltypeout{}%
\makeatother
\setcounter{page}{1}
\onecolumn

% !!! IMAGES START HERE !!!

\stepcounter{section}
\stepcounter{subsection}
\stepcounter{subsection}
\stepcounter{section}
{\newpage\clearpage
\lthtmlfigureA{tableii21}%
\begin{tableii}{c|l}{}{Hash function}{Digest length}
\lineii{MD2}{128 bits}
\lineii{MD4}{128 bits}
\lineii{MD5}{128 bits}
\lineii{RIPEMD}{160 bits}
\lineii{SHA1}{160 bits}
\lineii{SHA256}{256 bits}
\end{tableii}%
\lthtmlfigureZ
\lthtmlcheckvsize\clearpage}

{\newpage\clearpage
\lthtmlfigureA{datadesc44}%
\begin{datadesc}{digest_size}
An integer value; the size of the digest
produced by the hashing objects.  You could also obtain this value by
creating a sample object, and taking the length of the digest string
it returns, but using \member{digest_size} is faster.
\end{datadesc}%
\lthtmlfigureZ
\lthtmlcheckvsize\clearpage}

{\newpage\clearpage
\lthtmlfigureA{methoddesc48}%
\begin{methoddesc}{copy}{}
Return a separate copy of this hashing object.  An \code{update} to
this copy won't affect the original object.
\end{methoddesc}%
\lthtmlfigureZ
\lthtmlcheckvsize\clearpage}

{\newpage\clearpage
\lthtmlfigureA{methoddesc53}%
\begin{methoddesc}{digest}{}
Return the hash value of this hashing object, as a string containing
8-bit data.  The object is not altered in any way by this function;
you can continue updating the object after calling this function.
\end{methoddesc}%
\lthtmlfigureZ
\lthtmlcheckvsize\clearpage}

{\newpage\clearpage
\lthtmlfigureA{methoddesc57}%
\begin{methoddesc}{hexdigest}{}
Return the hash value of this hashing object, as a string containing
the digest data as hexadecimal digits.  The resulting string will be
twice as long as that returned by \method{digest()}.  The object is not
altered in any way by this function; you can continue updating the
object after calling this function.
\end{methoddesc}%
\lthtmlfigureZ
\lthtmlcheckvsize\clearpage}

{\newpage\clearpage
\lthtmlfigureA{methoddesc62}%
\begin{methoddesc}{update}{arg}
Update this hashing object with the string \var{arg}.
\end{methoddesc}%
\lthtmlfigureZ
\lthtmlcheckvsize\clearpage}

\stepcounter{subsection}
{\newpage\clearpage
\lthtmlinlinemathA{tex2html_wrap_inline518}%
$10" and "I owe
Bob $%
\lthtmlinlinemathZ
\lthtmlcheckvsize\clearpage}

\stepcounter{subsection}
\stepcounter{section}
{\newpage\clearpage
\lthtmlfigureA{tableii83}%
\begin{tableii}{c|l}{}{Cipher}{Key Size/Block Size}
\lineii{AES}{16, 24, or 32 bytes/16 bytes}
\lineii{ARC2}{Variable/8 bytes}
\lineii{Blowfish}{Variable/8 bytes}
\lineii{CAST}{Variable/8 bytes}
\lineii{DES}{8 bytes/8 bytes}
\lineii{DES3 (Triple DES)}{16 bytes/8 bytes}
\lineii{IDEA}{16 bytes/8 bytes}
\lineii{RC5}{Variable/8 bytes}
\end{tableii}%
\lthtmlfigureZ
\lthtmlcheckvsize\clearpage}

{\newpage\clearpage
\lthtmlfigureA{tableii106}%
\begin{tableii}{c|l}{}{Cipher}{Key Size}
\lineii{Cipher}{Key Size}
  \lineii{ARC4}{Variable}
  \lineii{XOR}{Variable}
\end{tableii}%
\lthtmlfigureZ
\lthtmlcheckvsize\clearpage}

{\newpage\clearpage
\lthtmlfigureA{funcdesc121}%
\begin{funcdesc}{new}{key, mode\optional{, IV}}
Returns a ciphering object, using \var{key} and feedback mode
\var{mode}.  If \var{mode} is \constant{MODE_CBC} or \constant{MODE_CFB}, \var{IV} must be provided,
and must be a string of the same length as the block size.  Some
algorithms support additional keyword arguments to this function; see
the "Algorithm-specific Notes for Encryption Algorithms" section below for the details.
\end{funcdesc}%
\lthtmlfigureZ
\lthtmlcheckvsize\clearpage}

{\newpage\clearpage
\lthtmlfigureA{datadesc131}%
\begin{datadesc}{block_size}
An integer value; the size of the blocks encrypted by this module.
Strings passed to the \code{encrypt} and \code{decrypt} functions
must be a multiple of this length.  For stream ciphers,
\code{block_size} will be 1. 
\end{datadesc}%
\lthtmlfigureZ
\lthtmlcheckvsize\clearpage}

{\newpage\clearpage
\lthtmlfigureA{datadesc137}%
\begin{datadesc}{key_size}
An integer value; the size of the keys required by this module.  If
\code{key_size} is zero, then the algorithm accepts arbitrary-length
keys.  You cannot pass a key of length 0 (that is, the null string
\code{''} as such a variable-length key.  
\end{datadesc}%
\lthtmlfigureZ
\lthtmlcheckvsize\clearpage}

{\newpage\clearpage
\lthtmlfigureA{memberdesc142}%
\begin{memberdesc}{block_size}
An integer value equal to the size of the blocks encrypted by this object.
Identical to the module variable of the same name.
\end{memberdesc}%
\lthtmlfigureZ
\lthtmlcheckvsize\clearpage}

{\newpage\clearpage
\lthtmlfigureA{memberdesc145}%
\begin{memberdesc}{IV}
Contains the initial value which will be used to start a cipher
feedback mode.  After encrypting or decrypting a string, this value
will reflect the modified feedback text; it will always be one block
in length.  It is read-only, and cannot be assigned a new value.
\end{memberdesc}%
\lthtmlfigureZ
\lthtmlcheckvsize\clearpage}

{\newpage\clearpage
\lthtmlfigureA{memberdesc148}%
\begin{memberdesc}{key_size}
An integer value equal to the size of the keys used by this object.  If
\code{key_size} is zero, then the algorithm accepts arbitrary-length
keys.  For algorithms that support variable length keys, this will be 0.
Identical to the module variable of the same name.  
\end{memberdesc}%
\lthtmlfigureZ
\lthtmlcheckvsize\clearpage}

{\newpage\clearpage
\lthtmlfigureA{methoddesc152}%
\begin{methoddesc}{decrypt}{string}
Decrypts \var{string}, using the key-dependent data in the object, and
with the appropriate feedback mode.  The string's length must be an exact
multiple of the algorithm's block size.  Returns a string containing
the plaintext.
\end{methoddesc}%
\lthtmlfigureZ
\lthtmlcheckvsize\clearpage}

{\newpage\clearpage
\lthtmlfigureA{methoddesc157}%
\begin{methoddesc}{encrypt}{string}
Encrypts a non-null \var{string}, using the key-dependent data in the
object, and with the appropriate feedback mode.  The string's length
must be an exact multiple of the algorithm's block size; for stream
ciphers, the string can be of any length.  Returns a string containing
the ciphertext.
\end{methoddesc}%
\lthtmlfigureZ
\lthtmlcheckvsize\clearpage}

\stepcounter{subsection}
\stepcounter{subsection}
\stepcounter{subsection}
\stepcounter{section}
\stepcounter{subsection}
{\newpage\clearpage
\lthtmlfigureA{classdesc180}%
\begin{classdesc}{AllOrNothing}{ciphermodule, mode=None, IV=None}
Class implementing the All-or-Nothing package transform.
\par
\var{ciphermodule} is a module implementing the cipher algorithm to
use.  Optional arguments \var{mode} and \var{IV} are passed directly
through to the \var{ciphermodule}.\code{new()} method; they are the
feedback mode and initialization vector to use.  All three arguments
must be the same for the object used to create the digest, and to
undigest'ify the message blocks.
\par
The module passed as \var{ciphermodule} must provide the \pep{272}
interface.  An encryption key is randomly generated automatically when
needed.
\end{classdesc}%
\lthtmlfigureZ
\lthtmlcheckvsize\clearpage}

{\newpage\clearpage
\lthtmlfigureA{methoddesc192}%
\begin{methoddesc}{digest}{text}
Perform the All-or-Nothing package transform on the 
string \var{text}.  Output is a list of message blocks describing the
transformed text, where each block is a string of bit length equal
to the cipher module's block_size.
\end{methoddesc}%
\lthtmlfigureZ
\lthtmlcheckvsize\clearpage}

{\newpage\clearpage
\lthtmlfigureA{methoddesc197}%
\begin{methoddesc}{undigest}{mblocks}
Perform the reverse package transformation on a list of message
blocks.  Note that the cipher module used for both transformations
must be the same.  \var{mblocks} is a list of strings of bit length
equal to \var{ciphermodule}'s block_size.  The output is a string object.
\end{methoddesc}%
\lthtmlfigureZ
\lthtmlcheckvsize\clearpage}

\stepcounter{subsection}
{\newpage\clearpage
\lthtmlfigureA{classdesc204}%
\begin{classdesc}{Chaff}{factor=1.0, blocksper=1}
Class implementing the chaff adding algorithm. 
\var{factor} is the number of message blocks 
            to add chaff to, expressed as a percentage between 0.0 and 1.0; the default value is 1.0.
\var{blocksper} is the number of chaff blocks to include for each block
            being chaffed, and defaults to 1.  The default settings 
add one chaff block to every
            message block.  By changing the defaults, you can adjust how
            computationally difficult it could be for an adversary to
            brute-force crack the message.  The difficulty is expressed as:
\par
\begin{verbatim}

pow(blocksper, int(factor * number-of-blocks))\end{verbatim}

\par
For ease of implementation, when \var{factor} < 1.0, only the first
\code{int(\var{factor}*number-of-blocks)} message blocks are chaffed.
\end{classdesc}%
\lthtmlfigureZ
\lthtmlcheckvsize\clearpage}

{\newpage\clearpage
\lthtmlfigureA{methoddesc215}%
\begin{methoddesc}{chaff}{blocks}
Add chaff to message blocks.  \var{blocks} is a list of 3-tuples of the
form (\var{serial-number}, \var{data}, \var{MAC}).
\par
Chaff is created by choosing a random number of the same
byte-length as \var{data}, and another random number of the same
byte-length as \var{MAC}.  The message block's serial number is placed
on the chaff block and all the packet's chaff blocks are randomly
interspersed with the single wheat block.  This method then
returns a list of 3-tuples of the same form.  Chaffed blocks will
contain multiple instances of 3-tuples with the same serial
number, but the only way to figure out which blocks are wheat and
which are chaff is to perform the MAC hash and compare values.
\end{methoddesc}%
\lthtmlfigureZ
\lthtmlcheckvsize\clearpage}

\stepcounter{section}
{\newpage\clearpage
\lthtmlfigureA{tableii228}%
\begin{tableii}{c|l}{}{Algorithm}{Capabilities}
\lineii{RSA}{Encryption, authentication/signatures}
\lineii{ElGamal}{Encryption, authentication/signatures}
\lineii{DSA}{Authentication/signatures}
\lineii{qNEW}{Authentication/signatures}
\end{tableii}%
\lthtmlfigureZ
\lthtmlcheckvsize\clearpage}

{\newpage\clearpage
\lthtmlfigureA{funcdesc244}%
\begin{funcdesc}{construct}{tuple}
Constructs a key object from a tuple of data.  This is
algorithm-specific; look at the source code for the details.  (To be
documented later.)
\end{funcdesc}%
\lthtmlfigureZ
\lthtmlcheckvsize\clearpage}

{\newpage\clearpage
\lthtmlfigureA{funcdesc248}%
\begin{funcdesc}{generate}{size, randfunc, progress_func=\code{None}}
Generate a fresh public/private key pair.  \var{size} is a
algorithm-dependent size parameter, usually measured in bits; the
larger it is, the more difficult it will be to break the key.  Safe
key sizes vary from algorithm to algorithm; you'll have to research
the question and decide on a suitable key size for your application.
An N-bit keys can encrypt messages up to N-1 bits long.
\par
\var{randfunc} is a random number generation function; it should
accept a single integer \var{N} and return a string of random data
\var{N} bytes long.  You should always use a cryptographically secure
random number generator, such as the one defined in the
\module{Crypto.Util.randpool} module; \emph{don't} just use the
current time and the \module{random} module. 
\par
\var{progress_func} is an optional function that will be called with a short
string containing the key parameter currently being generated; it's
useful for interactive applications where a user is waiting for a key
to be generated.
\end{funcdesc}%
\lthtmlfigureZ
\lthtmlcheckvsize\clearpage}

{\newpage\clearpage
\lthtmlfigureA{methoddesc261}%
\begin{methoddesc}{can_blind}{}
Returns true if the algorithm is capable of blinding data; 
returns false otherwise.  
\end{methoddesc}%
\lthtmlfigureZ
\lthtmlcheckvsize\clearpage}

{\newpage\clearpage
\lthtmlfigureA{methoddesc265}%
\begin{methoddesc}{can_encrypt}{}
Returns true if the algorithm is capable of encrypting and decrypting
data; returns false otherwise.  To test if a given key object can encrypt
data, use \code{key.can_encrypt() and key.has_private()}.
\end{methoddesc}%
\lthtmlfigureZ
\lthtmlcheckvsize\clearpage}

{\newpage\clearpage
\lthtmlfigureA{methoddesc270}%
\begin{methoddesc}{can_sign}{}
Returns true if the algorithm is capable of signing data; returns false
otherwise.  To test if a given key object can sign data, use
\code{key.can_sign() and key.has_private()}.
\end{methoddesc}%
\lthtmlfigureZ
\lthtmlcheckvsize\clearpage}

{\newpage\clearpage
\lthtmlfigureA{methoddesc275}%
\begin{methoddesc}{decrypt}{tuple}
Decrypts \var{tuple} with the private key, returning another string.
This requires the private key to be present, and will raise an exception
if it isn't present.  It will also raise an exception if \var{string} is
too long.
\end{methoddesc}%
\lthtmlfigureZ
\lthtmlcheckvsize\clearpage}

{\newpage\clearpage
\lthtmlfigureA{methoddesc281}%
\begin{methoddesc}{encrypt}{string, K}
Encrypts \var{string} with the private key, returning a tuple of
strings; the length of the tuple varies from algorithm to algorithm.  
\var{K} should be a string of random data that is as long as
possible.  Encryption does not require the private key to be present
inside the key object.  It will raise an exception if \var{string} is
too long.  For ElGamal objects, the value of \var{K} expressed as a
big-endian integer must be relatively prime to \code{self.p-1}; an
exception is raised if it is not.
\end{methoddesc}%
\lthtmlfigureZ
\lthtmlcheckvsize\clearpage}

{\newpage\clearpage
\lthtmlfigureA{methoddesc290}%
\begin{methoddesc}{has_private}{}
Returns true if the key object contains the private key data, which
will allow decrypting data and generating signatures.
Otherwise this returns false.
\end{methoddesc}%
\lthtmlfigureZ
\lthtmlcheckvsize\clearpage}

{\newpage\clearpage
\lthtmlfigureA{methoddesc294}%
\begin{methoddesc}{publickey}{}
Returns a new public key object that doesn't contain the private key
data. 
\end{methoddesc}%
\lthtmlfigureZ
\lthtmlcheckvsize\clearpage}

{\newpage\clearpage
\lthtmlfigureA{methoddesc298}%
\begin{methoddesc}{sign}{string, K}
Sign \var{string}, returning a signature, which is just a tuple; in
theory the signature may be made up of any Python objects at all; in
practice they'll be either strings or numbers.  \var{K} should be a
string of random data that is as long as possible.  Different algorithms
will return tuples of different sizes.  \code{sign()} raises an
exception if \var{string} is too long.  For ElGamal objects, the value
of \var{K} expressed as a big-endian integer must be relatively prime to
\code{self.p-1}; an exception is raised if it is not.
\end{methoddesc}%
\lthtmlfigureZ
\lthtmlcheckvsize\clearpage}

{\newpage\clearpage
\lthtmlfigureA{methoddesc308}%
\begin{methoddesc}{size}{}
Returns the maximum size of a string that can be encrypted or signed,
measured in bits.  String data is treated in big-endian format; the most
significant byte comes first.  (This seems to be a \emph{de facto} standard
for cryptographical software.)  If the size is not a multiple of 8, then
some of the high order bits of the first byte must be zero.  Usually
it's simplest to just divide the size by 8 and round down.
\end{methoddesc}%
\lthtmlfigureZ
\lthtmlcheckvsize\clearpage}

{\newpage\clearpage
\lthtmlfigureA{methoddesc313}%
\begin{methoddesc}{verify}{string, signature}
Returns true if the signature is valid, and false otherwise.
\var{string} is not processed in any way; \code{verify} does
not run a hash function over the data, but you can easily do that yourself.
\end{methoddesc}%
\lthtmlfigureZ
\lthtmlcheckvsize\clearpage}

\stepcounter{subsection}
\stepcounter{subsection}
\stepcounter{section}
\stepcounter{subsection}
{\newpage\clearpage
\lthtmlfigureA{funcdesc341}%
\begin{funcdesc}{GCD}{x,y}
Return the greatest common divisor of \var{x} and \var{y}.
\end{funcdesc}%
\lthtmlfigureZ
\lthtmlcheckvsize\clearpage}

{\newpage\clearpage
\lthtmlfigureA{funcdesc347}%
\begin{funcdesc}{getPrime}{N, randfunc}
Return an \var{N}-bit random prime number, using random data obtained
from the function \var{randfunc}.  \var{randfunc} must take a single
integer argument, and return a string of random data of the
corresponding length; the \method{get_bytes()} method of a
\class{RandomPool} object will serve the purpose nicely, as will the
\method{read()} method of an opened file such as \file{/dev/random}.
\end{funcdesc}%
\lthtmlfigureZ
\lthtmlcheckvsize\clearpage}

{\newpage\clearpage
\lthtmlfigureA{funcdesc358}%
\begin{funcdesc}{getRandomNumber}{N, randfunc}
Return an \var{N}-bit random number, using random data obtained from the
function \var{randfunc}.  As usual, \var{randfunc} must take a single
integer argument and return a string of random data of the
corresponding length.
\end{funcdesc}%
\lthtmlfigureZ
\lthtmlcheckvsize\clearpage}

{\newpage\clearpage
\lthtmlfigureA{funcdesc365}%
\begin{funcdesc}{inverse}{u, v}
Return the inverse of \var{u} modulo \var{v}.
\end{funcdesc}%
\lthtmlfigureZ
\lthtmlcheckvsize\clearpage}

{\newpage\clearpage
\lthtmlfigureA{funcdesc371}%
\begin{funcdesc}{isPrime}{N}
Returns true if the number \var{N} is prime, as determined by a
Rabin-Miller test.
\end{funcdesc}%
\lthtmlfigureZ
\lthtmlcheckvsize\clearpage}

\stepcounter{subsection}
{\newpage\clearpage
\lthtmlfigureA{classdesc388}%
\begin{classdesc}{RandomPool}{\optional{numbytes, cipher, hash} }
An object of the \code{RandomPool} class can be created without
parameters if desired.  \var{numbytes} sets the number of bytes of
random data in the pool, and defaults to 160 (1280 bits). \var{hash}
can be a string containing the module name of the hash function to use
in stirring the random data, or a module object supporting the hashing
interface.  The default action is to use SHA.
\par
The \var{cipher} argument is vestigial; it was removed from version
1.1 so RandomPool would work even in the limited exportable subset of
the code.  I recommend passing \var{hash} using a keyword argument so
that someday I can safely delete the \var{cipher} argument
\par
\end{classdesc}%
\lthtmlfigureZ
\lthtmlcheckvsize\clearpage}

{\newpage\clearpage
\lthtmlfigureA{methoddesc399}%
\begin{methoddesc}{add_event}{time\optional{, string}}
Adds an event to the random pool.  \var{time} should be set to the
current system time, measured at the highest resolution available.
\var{string} can be a string of data that will be XORed into the pool,
and can be used to increase the entropy of the pool.  For example, if
you're encrypting a document, you might use the hash value of the
document; an adversary presumably won't have the plaintext of the
document, and thus won't be able to use this information to break the
generator.
\end{methoddesc}%
\lthtmlfigureZ
\lthtmlcheckvsize\clearpage}

{\newpage\clearpage
\lthtmlfigureA{memberdesc409}%
\begin{memberdesc}{bits}
A constant integer value containing the number of bits of data in
the pool, equal to the \member{bytes} attribute multiplied by 8.
\end{memberdesc}%
\lthtmlfigureZ
\lthtmlcheckvsize\clearpage}

{\newpage\clearpage
\lthtmlfigureA{memberdesc413}%
\begin{memberdesc}{bytes}
A constant integer value containing the number of bytes of data in
the pool.
\end{memberdesc}%
\lthtmlfigureZ
\lthtmlcheckvsize\clearpage}

{\newpage\clearpage
\lthtmlfigureA{memberdesc416}%
\begin{memberdesc}{entropy}
An integer value containing the number of bits of entropy currently in
the pool.  The value is incremented by the \method{add_event()} method,
and decreased by the \method{get_bytes()} method.
\end{memberdesc}%
\lthtmlfigureZ
\lthtmlcheckvsize\clearpage}

{\newpage\clearpage
\lthtmlfigureA{methoddesc421}%
\begin{methoddesc}{get_bytes}{num}
Returns a string containing \var{num} bytes of random data, and
decrements the amount of entropy available.  It is not an error to
reduce the entropy to zero, or to call this function when the entropy
is zero.  This simply means that, in theory, enough random information has been
extracted to derive the state of the generator.  It is the caller's
responsibility to monitor the amount of entropy remaining and decide
whether it is sufficent for secure operation.
\end{methoddesc}%
\lthtmlfigureZ
\lthtmlcheckvsize\clearpage}

{\newpage\clearpage
\lthtmlfigureA{methoddesc426}%
\begin{methoddesc}{stir}{}
Scrambles the random pool using the previously chosen encryption and
hash function.  An adversary may attempt to learn or alter the state
of the pool in order to affect its future output; this function
destroys the existing state of the pool in a non-reversible way.  It
is recommended that \method{stir()} be called before and after using
the \class{RandomPool} object.  Even better, several calls to
\method{stir()} can be interleaved with calls to \method{add_event()}.
\end{methoddesc}%
\lthtmlfigureZ
\lthtmlcheckvsize\clearpage}

{\newpage\clearpage
\lthtmlfigureA{classdesc436}%
\begin{classdesc}{PersistentRandomPool}{filename, \optional{numbytes, cipher, hash}}
The path given in \var{filename} will be automatically opened, and an
existing random pool read; if no such file exists, the pool will be
initialized as usual.  If omitted, the filename defaults to the empty
string, which will prevent it from being saved to a file.  These
arguments are identical to those for the \class{RandomPool}
constructor.
\end{classdesc}%
\lthtmlfigureZ
\lthtmlcheckvsize\clearpage}

{\newpage\clearpage
\lthtmlfigureA{methoddesc442}%
\begin{methoddesc}{save}{}
Opens the file named by the \member{filename} attribute, and saves the
random data into the file using the \module{pickle} module.
\end{methoddesc}%
\lthtmlfigureZ
\lthtmlcheckvsize\clearpage}

{\newpage\clearpage
\lthtmlfigureA{methoddesc450}%
\begin{methoddesc}{randomize}{}
(Unix systems only)  Obtain random data from the keyboard.  This works
by prompting the
user to hit keys at random, and then using the keystroke timings (and
also the actual keys pressed) to add entropy to the pool.  This works
similarly to PGP's random pool mechanism.
\end{methoddesc}%
\lthtmlfigureZ
\lthtmlcheckvsize\clearpage}

\stepcounter{subsection}
{\newpage\clearpage
\lthtmlfigureA{funcdesc455}%
\begin{funcdesc}{key_to_english}{key}
Accepts a string of arbitrary data \var{key}, and returns a string
containing uppercase English words separated by spaces.  \var{key}'s
length must be a multiple of 8.
\end{funcdesc}%
\lthtmlfigureZ
\lthtmlcheckvsize\clearpage}

{\newpage\clearpage
\lthtmlfigureA{funcdesc461}%
\begin{funcdesc}{english_to_key}{string}
Accepts \var{string} containing English words, and returns a string of
binary data representing the key.  Words must be separated by
whitespace, and can be any mixture of uppercase and lowercase
characters.  6 words are required for 8 bytes of key data, so
the number of words in \var{string} must be a multiple of 6.
\end{funcdesc}%
\lthtmlfigureZ
\lthtmlcheckvsize\clearpage}

\stepcounter{section}
\stepcounter{subsection}
\stepcounter{subsection}
\stepcounter{subsection}

\end{document}
